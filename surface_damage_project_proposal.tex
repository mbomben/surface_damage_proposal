\documentclass[dvips,12pt]{article}

\usepackage[pdftex]{graphicx}
\usepackage{url}
\usepackage{enumerate}


\setlength{\oddsidemargin}{0.25in}
\setlength{\textwidth}{6.5in}
\setlength{\topmargin}{0in}
\setlength{\textheight}{8.5in}


\begin{document}


\title{Study of the Si-SiO$_2$~interface damage and its interplay with the bulk damage}
\author{M.~Bomben, A.~Ducourthial, A.~Morozzi,  F.~Moscatelli, D.~Passeri and J.~Zhang}
\date{\today}



\maketitle

\begin{abstract}
In silicon radiation sensors not only the bulk but also the Si-SiO$_2$~interface can be severely 
affected by radiation damage. This is true not only for sensors aimed at the HL-LHC detectors 
but also for those conceived to be used at X-ray free electron laser facilities like the European XFEL. 
In this project proposal we want to address this topic by combining a mixed irradiation program 
(photons and reactor neutrons) with TCAD simulations, the latter to be used to interpret the 
measurements performed on irradiated sensors. The final goal is to develop an effective model 
for n- and p-type bulk sensors for bulk and surface radiation damage to be used in TCAD 
simulations.

\end{abstract}


\section{Introduction}

During the High Luminosity phase of the LHC (HL-LHC)~\cite{HL-LHC} the innermost layers of 
 the trackers' experiments will have to 
withstand, integrated over 10 year of data-taking,
  fluences of $\Phi = $1-2$\times$10$^{16}{\rm n_{eq}/cm^2}$  and total ionising doses of the order 
  of  
 TID~=~1Grad. Experiments at the European X-ray Free Electron Laser (XFEL) require silicon 
 pixel sensors which can withstand X-ray doses up to 100 Grad~\cite{zhang-thesis}.
 
 Non ionising energy loss (NIEL) is the only relevant radiation damage effect for the 
 bulk of silicon radiation sensors; ionising energy loss is the dominant effect for what concerns 
 SiO$_2$ and Si-SiO$_2$~interface radiation damage.
 Both mechanisms introduce defects in the silicon forbidden bandgap; ionising energy loss is 
 also responsible for the increase of the fixed oxide charge. Both bulk and surface defects 
 can be charged, as they act as carrier traps; they can also act as generation sites. Hence, 
 both kind of defects determine a change of working 
 voltages\footnote{depletion voltage in diodes, flat band voltage in MOS structures}, a decrease 
 of the signal amplitude and an increase of leakage current, both bulk- and surface-generated.

In order to interpret the observed phenomena after detectors irradiations,
 and to make valid predictions for irradiated silicon 
sensors performances, an effective parametrisation of the radiation damage effects is needed. 
Several attempts have been made in the past, separately for bulk and surface damage. 
Recently an effort started to develop a radiation damage parametrisation that includes both 
surface and bulk damage~\cite{Passeri2015}, aimed at sensors designed for the HL-LHC 
phase.

An extensive study of surface radiation damage has been carried out for the sensors 
for Science at the XFEL; a detailed summary can be found at~\cite{zhang-thesis}.
Those studies deal with n-bulk material sensors; in this project we want to apply the 
same approach for those studies to p-bulk material sensors too.

The tools to verify the correctness of the new radiation damage model and then to make 
sound predictions are TCAD\footnote{Technology Computer Aided Design} simulations. 

In order to get reliable predictions from TCAD simulations, the latter have to be validated 
on data from irradiated samples. X-Rays will be considered in our project, as 
well as reactor neutrons.
Photon irradiation will create defects at the surface and will make the fixed oxide charge increase. 
Reactor neutrons will damage only the silicon bulk~\footnote{Yet we should account for the 
$\gamma$-ray background from the reactor. At JSI, about 1~Mrad of dose is expected 
at a fluence of $1\times10^{15} {\rm n_{eq}/cm^2}$}. In this way we can disentangle the two 
radiation damage mechanisms.  

In Section~\ref{sec:goals} the project goals will be presented. After having outlined the methods 
(Section~\ref{sec:methods}), in Section~\ref{sec:resources} the needed resources will be 
presented. The Section~\ref{sec:timeline} will present a tentative timeline for the project. 

\section{Project goals}
\label{sec:goals}
The following list contains the main project goals:
\begin{itemize}
\item Irradiate test structures 
\begin{enumerate}[a)]
\item  first with X-rays and then with 
\item reactor neutrons
\end{enumerate}
\item Perform measurements on the irradiated test structures to study parameters like: 
\begin{enumerate}[a)]
\item the rate of change of depletion voltage  as a function of the integrated fluence,
\item same for the leakage current
\item and for trapping, plus
\item flat-band voltage shift,
\item the increase of leakage current due to the surface damage and
\item the increase of the fixed oxide charge
\end{enumerate}
\item Analyse the results to extract a comprehensive defect parametrization accounting for combined 
substrate and surface damage on both n-type and p-type substrates.
\item Use TCAD simulations to 
\begin{enumerate}[a)] 
\item reproduce the results drawn from irradiated samples measurements and
\item make predictions for interesting scenarios like those expected at LH-LHC and XFEL
\end{enumerate}
\end{itemize}

Interesting test structures are: diodes, MOS capacitors, gated diodes, interstrip resistance and 
capacitance measurement dedicated structures.

\section{Methods}
\label{sec:methods}
To meet the project goals presented in Section~\ref{sec:goals} the following test structures 
and facilities are necessary, both n-type and p-type.
\paragraph{Test structures} As already mentioned, interesting test structures are: diodes, MOS 
capacitors, gated diodes, interstrip resistance- and 
capacitance-measurement dedicated structures.
Diodes will allow to extract bulk depletion voltage $V_{depl}$ and leakage current $I_{leak}$
 increase with fluence/dose. 
MOS capacitors give access to the value of fixed oxide charge $N_{ox}$ density and to the energy 
 $E_{it}$ and distribution  $D_{it}(E_{it})$ of effective surface defects. Gated diodes 
are crucial to measure the surface generated current density $J_{surf}$. 
Diodes can also be used to measure the radiation damage induced signal reduction and obtain 
a trapping time constant $\tau_{tr}$ estimate. Interstrip resistance- and 
capacitance-measurement dedicated structures are very useful to study surface properties 
of irradiated sensors like the fixed oxide charge $N_{ox}$ density.

\noindent Avoiding the presence of p-spray in test structures that were taken from
 n-on-p or n-on-n productions is crucial to perform measurements on MOS 
capacitors and gated diodes in an optimal way.
\noindent The authors have already some test structures in hand; those  were obtained from different 
FBK productions. 

\paragraph{Irradiation facilities}
X-rays irradiation facilities are sought for, and neutron reactors as well.
In the past similar studies were carried out at:
\begin{itemize}
\item x-ray tubes in Hamburg (FSDs) ({\it please provide some information here})
\item similar facility in Padua (low energy ($<$60~keV)) X-ray sources are currently used for total dose tests of unpackaged semiconductor devices and circuits due to the higher dose rates (100-1000 rad(Si)/s), lower cost and radioprotection constraints than $^{60}$Co $\gamma$-ray sources.
The X-ray facility installed at the Department of Physics and Astronomy of the University of Padova (Italy) is the Seifert RP-149 Semiconductor Irradiation System equipped with a standard tube for X-ray diffraction analysis: maximum power 3000 W, maximum voltage 60 kV, tungsten anode, 0.3 mm Be inherent filtration, take off angle 6 degree. 
\item synchrotron in DESY({\it please provide some information here})
\item TRIGA Mark II research reactor at the Jo\v{z}ef Stefan Institute (JSI)  
\end{itemize}
\noindent Sought doses and fluences:
\begin{itemize}
\item doses: 50krad, 200krad, 2Mrad, 5Mrad, 10Mrad, 100Mrad and 1Grad (for HL-LHC 
applications)
\item fluences: $1\times10^{13}$,  $3\times10^{13}$, $1\times10^{14}$, $3\times10^{14}$, 
$1\times10^{15}$, $3\times10^{15}$, $7.5\times10^{15}$, $1\times10^{16}$, $1.5\times10^{16}$ and 
$2\times10^{16}\,{\rm n_{eq}/cm^2}$.
\end{itemize}

\noindent The number of irradiations we will able to carry out will of course depend on the 
number of test structures we have and on the availability of the sources/facilities. Priorities 
will be identified among the aforementioned fluences and doses once number of test structures 
and  availability of the sources/facilities will be known. 
\noindent The very low fluences ($1\times10^{13}$,  
$3\times10^{13}$, $1\times10^{14}$) are intended for low-resistivity bulk material; the goal 
is to study the initial acceptor removal phenomenon. 

\paragraph{Measurement set-ups} At the moment four laboratories have been identified: Hamburg, CNR-IMM Bologna, LPNHE Paris and CERN.

\subparagraph{On diodes} I-V and C-V measurements will be performed; these 
measurements are possible with Semiconductor Device Parameter Analyzer or 
with even simpler systems; such tools are available in Hamburg, CNR-IMM Bologna, in LPNHE Paris
and at CERN. 

\subparagraph{On gated diodes} I-V curves can be measured at the same laboratories
 already mentioned.

\subparagraph{On MOS capacitors}  C/G-V measurements can be performed in Hamburg, CNR-
IMM Bologna, in  LPNHE Paris and at CERN. Thermal Dielectric Relaxation Current (TDRC) 
measurement allows to extract the energy 
 $E_{it}$ and distribution  $D_{it}(E_{it})$ of effective surface defects; this kind of measurement 
 is possible in Hamburg.

\subparagraph{On  interstrip resistance- and capacitance-measurement dedicated test structures} 
the interstrip isolation  can be studied  using 
Semiconductor Device Parameter Analyzer or with even simpler systems available in Hamburg, CNR-IMM Bologna, in LPNHE Paris
and at CERN. 

\noindent In all cases below room temperature set-ups are available.

\paragraph{TCAD simulation tools}The plan is to use TCAD 2D and 3D simulation tools 
to reproduce the measurement data. LPNHE Paris and University of Perugia have access to 
Silvaco and Synopsys TCAD simulation tools, respectively. 

\section{Resources requests}

\paragraph{Test structures}More test structures are needed. The proponents are already 
in contact with the following foundries/groupes: FBK, KEK/HPK and Micron.
\paragraph{Irradiation facilities}The proponents will apply for AIDA-2020 support for irradiation 
campaign at TRIGA reactor. Access to Hamburg and Padua source is under investigation.
\paragraph{Measurement facilities} Access to Hamburg measurement facilities (TDRC in 
particular) has to be negotiated. Access to CERN Silicon Lab will be asked. 
Financial support for travel expenses would be a plus; two 
PhD students are involved in this project.
\paragraph{TCAD simulations}Both LPNHE Paris and University of Perugia have access to 
TCAD simulation tools so in principle no further support is needed here.


\label{sec:resources}

\section{Timeline}
\label{sec:timeline}
{\it very tentative}

\begin{enumerate}[1)]
\item File the proposal well before the next RD50 meeting (foreseen on 6-8 June); let's say end of 
April.
\item Meanwhile survey the market for more test structures: reiterate with Micron. 
Have a first list of replies (and available structures) by mid-May.
\item Start the discussion with Hamburg group by presenting them a draft of the project proposal, 
after having included the list of samples of the previous point; late May.
\item Present the project at the RD50 meeting.
\item Get approved and supported by the RD50 collaboration - I am not sure about the official 
procedure, that's why I am in favour of filing a draft to Micheal and Gianluigi to get advice and know 
the technicalities.
\item Start doing pre-irradiation measurements - from mid June onward.
\item First irradiations - late June, July
\item First measurements of irradiated detectors - if possible start in July
\item First simulations - July
\end{enumerate}

\newpage
%At this point I won't progress any further due to the limited amount of information we have at the 
%moment.

\bibliography{biblio}
\bibliographystyle{IEEEtran}


\end{document}
